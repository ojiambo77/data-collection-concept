

\documentclass[12pt,]{article}
\usepackage{zed-csp,graphicx,color}%from
\pagenumbering{roman}
\begin{document}

\begin{titlepage}
\centerline{Establishment of a birtdate recording system to do away with paper work in hospitals of Uganda.\\}
\paragraph*{•}
\centerline{  Prepared by:  Ojiambo Abex 16/u/10829/ps 216013324.\\}
\paragraph*{•}
\paragraph*{•}
  \begin{flushright}
  The Report,\\
  DATE: $February,25^{th},2018$.
 \tableofcontents

  \end{flushright}
\date{\today}
\end{titlepage}

\newpage





\pagenumbering{arabic}
\section{Introduction}
Birthdate recording system is a system majorly to be used in hospitals to record the date of birth of newly born children on a particular day to solve paper writing in hospitals. This in turn saves time and reduces wastage of resources such as papers which may get torn as time goes on hence losing the details of the baby.
\section{Statement of the problem}
Paper writing is still more in hospitals because of limited resources for implementation of the system and limited skilled labour in accordance to technology .
\section{Main objectives of the study}
To find a way of eliminating paper writing in hospitals and ensure safety of records.
\section{specific objective of the study}

To ensure citizenship of children in the country as their bio-data is recorded.
To easily know the total number of children born in a country on a particular day.
To  ensure that every children in a country has a known age and the date of birth.


\section{Scope of the study}
Due to the above problems I had to visit Dr. Tumusirye Annet of Mulago hosipital and we had a face to face interview where she elaborated the paper writing problem in Mulago . In this she said many people have come along claiming for the loss of their children yet their birth certificates we would use to search for the child following the bio-data are nowhere to be seen .when some parents are asked for the certificates, they say  it fall in water and others are like they were eaten by rats in the house. She concluded by saying we should develop for them a date of birth recording system.

\section{Methodology}
The data will be obtained using the ODK collect and it will later be uploaded on the ODK aggregate server.
This was done with Dr. Tumusirye Annet of mulago hospital and later went to Lubaga hospital where we spoke with Dr. Kigundu Denis.
\section{Recommendations }
Based on the findings and conclusions in this study, the following recommendations are made:

\begin{enumerate}

\item Hospital should adopted a well trained information technology staff to implement a birthday recording system to allow recording of children bio-data on machines hence increase on the safety of data.






\end{enumerate}

\end{document}






